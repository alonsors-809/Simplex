% Technological of Costa Rica 
% Operations Research
% 3rd Project
% Linear Programming
% Alonso Rivas Solano (2014079916)
% Daniel Herrera Brenes (2015130539)
% Edisson López Díaz (2013103311)
 
\documentclass{beamer} 
\usetheme[progressbar=frametitle]{metropolis} 
\setbeamertemplate{frame numbering}[fraction] 
\useoutertheme{metropolis} 
\useinnertheme{metropolis} 
\usefonttheme{metropolis} 
\usecolortheme{metropolis} 
\usepackage[utf8]{inputenc} 
\usepackage{lmodern} 
\usepackage[T1]{fontenc} 
\usepackage[spanish]{babel} 
\usepackage{tikz} 
\usepackage{natbib} 
\usepackage{hyperref} 
\usepackage{multirow} 
\usepackage{colortbl} 
\usepackage{helvet} 
\usepackage[export]{adjustbox} % loads also graphicx 
\usepackage{lipsum} 
%Definiciones 
\definecolor{color_columna_candidata}{rgb}{0, 0.424, 0.455} 
\definecolor{color_pivote}{rgb}{0.973, 0.80, 0.341} 
\definecolor{color_blanco}{rgb}{1,1,1} 
% Commands 
\newcommand\tab[1][1cm]{\hspace*{#1}}  
\newcommand\minitab[1][0.5cm]{\hspace*{#1}}  
% Tittle information 
\title{Simplex} 
\subtitle{Operations Research} 
\author[A. \& D. \& E.]{% 
\texorpdfstring{% 
\begin{columns} 
\column{.33\linewidth} 
\centering 
\\  Daniel Herrera  \\ 2015130539 \\ 
\column{.33\linewidth} 
\centering 
\\  Edisson López \\ 2013103311 \\ 
\column{.33\linewidth} 
\centering 
\\ Alonso Rivas \\ 2014079916 \\ 
\end{columns} 
} 
{Author 1, Author 2, Author 3} 
} 
\date{} 
\institute{% 
\texorpdfstring{% 
\begin{columns} 
\column{.9\linewidth} 
\centering 
\\ 
Tecnológico de Costa Rica \\ 
Semester 1, 2018 \\ 
May 24, 2018 
\end{columns} 
} 
} 
%Inicio del documento 
\begin{document} 

% - - 1st Slide - - ; 
% - - Cover - - - - ; 
\begin{frame}[plain,t] 
\maketitle 
\end{frame} 


% - - - - - - - - - ;
% - - - - 2 - - - - ;
% Algoritmo Símplex: uno o dos slides que expliquen
% un poco el algoritmo Sı́mplex.
\section{Simplex Algorithm}
\begin{frame}
The simplex is a method to solve lineal programming problems. This is a mechanical method that search for the best or optimal solution for a lineal programming(LP) problem. It was invented by George Danzig in 1947. It uses operations over a matrix to search for the optimal solution. It begin from a feasible region and it starts to do some operations, depending if you are maximizing or minimizing that search for the candidate column and the pivot, and after all the numbers are positive or negative, depends if maximizing or minimizing, that it give you the best solution.
\end{frame}

 
\section{Original Problem}  
\begin{frame}[shrink]  
\frametitle{Agricultura
} 
\begin{alertblock}{Maximize} 
\begin{itemize} 
\item $Z = -10000 x1 - 3000 x2$ 
\end{itemize} 
\end{alertblock} 
\begin{alertblock}{Constraints} 
\begin{enumerate} 
\item $ 1x1  + 1 x2 \leq 4$ 
\item $ 10x1  + 4 x2 \leq 10$ 
\item $ 0x1  + 1 x2 \geq 3$ 
\end{enumerate} 
\end{alertblock} 
\end{frame} 

\section{Initial Table} 
 
\begin{frame}  
\frametitle{Initial Table} 
\begin{table}[H] 
\begin{center} 
\resizebox{\linewidth}{!}{ 
\begin{tabular}{|*{8}{c|}} 
\hline 
\textbf{Z}  & \textbf{x1} & \textbf{x2} & \textbf{s$_{1}$} & \textbf{s$_{2}$} & \textbf{e$_{1}$} & \textbf{a$_{1}$} & \textbf{•} \\\hline \hline 
1 & -10000 & -3000 & 0 & 0 & 0 & 1M & 0 \\\hline 
0 & 1 & 1 & 1 & 0 & 0 & 0 & 4\\ 
\hline 
0 & 10 & 4 & 0 & 1 & 0 & 0 & 10\\ 
\hline 
0 & 0 & 1 & 0 & 0 & -1 & 1 & 3\\ 
\hline 
\end{tabular}} 
\caption{Initial Table.} 
\end{center} 
\end{table} 
\end{frame} 
 
\section{Intermediates Tables} 
 
\begin{frame}  
\frametitle{Intermediate Table 1} 
\begin{table}[H] 
\begin{center} 
\resizebox{\linewidth}{!}{ 
\begin{tabular}{|*{8}{c|}} 
\hline 
\textbf{Z}  & \textbf{x1} & \textbf{x2} & \textbf{s$_{1}$} & \textbf{s$_{2}$} & \textbf{e$_{1}$} & \textbf{•} \\\hline \hline 
1 & -10000 & -3000-1M & 0 & 0 & 1M & -3M \\\hline 
0 & 1 & 1 & 1 & 0 & 0 & 4\\ 
\hline 
0 & 10 & 4 & 0 & 1 & 0 & 10\\ 
\hline 
0 & 0 & 1 & 0 & 0 & -1 & 3\\ 
\hline 
\end{tabular}} 
\caption{Intermediate Table 1, with the column a$_{1}$ canonized.} 
\end{center} 
\end{table} 
\end{frame} 
 
 
\begin{frame}  
\frametitle{Intermediate Table \#1} 
\begin{table}[H] 
\begin{center} 
\resizebox{\linewidth}{!}{ 
\begin{tabular}{|*{8}{c|}} 
\hline 
\textbf{Z}  & \textbf{x1} & \cellcolor{color_columna_candidata}\textcolor{color_blanco}{\textbf{x2}} & \textbf{s$_{1}$} & \textbf{s$_{2}$} & \textbf{e$_{1}$} & \textbf{•} \\\hline \hline 
1 & -10000 & \cellcolor{color_columna_candidata}\textcolor{color_blanco}{-3000-1M} & 0 & 0 & 1M & -3M \\\hline 
0 & 1 & \cellcolor{color_columna_candidata}\textcolor{color_blanco}{1} & 1 & 0 & 0 & 4\\ 
\hline 
0 & 10 & \cellcolor{color_pivote}\textbf{4} & 0 & 1 & 0 & 10\\ 
\hline 
0 & 0 & \cellcolor{color_columna_candidata}\textcolor{color_blanco}{1} & 0 & 0 & -1 & 3\\ 
\hline 
\end{tabular}} 
\caption{Intermediate Table 1, during the pivoteo.} 
{\scriptsize Calculations: 4/1 = 4 | \textbf{10/4 = 2,5} | 3/1 = 3} 
\end{center} 
\end{table} 
\end{frame} 
 
 
\begin{frame}  
\frametitle{Intermediate Table \#2} 
\begin{table}[H] 
\begin{center} 
\resizebox{\linewidth}{!}{ 
\begin{tabular}{|*{8}{c|}} 
\hline 
\textbf{Z}  & \textbf{x1} & \cellcolor{color_columna_candidata}\textcolor{color_blanco}{\textbf{x2}} & \textbf{s$_{1}$} & \textbf{s$_{2}$} & \textbf{e$_{1}$} & \textbf{•} \\\hline \hline 
1 & -2500+2,5M & \cellcolor{color_columna_candidata}\textcolor{color_blanco}{0} & 0 & 750+0,2M & 1M & 7500+-0,5M \\\hline 
0 & -1,5 & \cellcolor{color_columna_candidata}\textcolor{color_blanco}{0} & 1 & -0,2 & 0 & 1,5\\ 
\hline 
0 & 2,5 & \cellcolor{color_columna_candidata}\textcolor{color_blanco}{1} & 0 & 0,2 & 0 & 2,5\\ 
\hline 
0 & -2,5 & \cellcolor{color_columna_candidata}\textcolor{color_blanco}{0} & 0 & -0,2 & -1 & 0,5\\ 
\hline 
\end{tabular}} 
\caption{Intermediate Table 2, with the column 3 canonized.} 
\end{center} 
\end{table} 
\end{frame} 
 
\section{Final Table} 
 
\begin{frame}  
\frametitle{Final Table} 
\begin{table}[H] 
\begin{center} 
\resizebox{\linewidth}{!}{ 
\begin{tabular}{|*{8}{c|}} 
\hline 
\textbf{Z}  & \textbf{x1} & \textbf{x2} & \textbf{s$_{1}$} & \textbf{s$_{2}$} & \textbf{e$_{1}$} & \textbf{a$_{1}$} & \textbf{•} \\\hline \hline 
1 & -2500+2,5M & 0 & 0 & 750+0,2M & 1M & 0 & 7500+-0,5M \\\hline 
0 & -1,5 & 0 & 1 & -0,2 & 0 & 0 & 1,5\\ 
\hline 
0 & 2,5 & 1 & 0 & 0,2 & 0 & 0 & 2,5\\ 
\hline 
0 & -2,5 & 0 & 0 & -0,2 & -1 & 1 & 0,5\\ 
\hline 
\end{tabular}} 
\caption{Final Table.} 
\end{center} 
\end{table} 
\end{frame} 
 

\section{Solution} 
\begin{frame} 
\frametitle{Solution} 
\begin{exampleblock}{Optimal solution} 
{\scriptsize Agricultura
} 
\begin{itemize} 
\item $Z = 7500$ 
\item $x_{2} = 2,5$ 
\end{itemize} 
\end{exampleblock} 
\end{frame} 


\begin{frame} 
\frametitle{Especial Cases} 
\begin{exampleblock}{} 
The problem had the following special cases: 
\begin{enumerate} 
\item Not Feasible Problem  
\end{enumerate} 
\end{exampleblock} 
In the following slides this will be explained. 
\end{frame} 


\begin{frame} 
\frametitle{Not Feasible Problem} 
The problem became not feasible when making a tour of the first row of the final table and found Ms.
\end{frame} 

\begin{frame}\frametitle{}\begin{center}{\Huge - final slide -}\end{center}\end{frame} 
\end{document}
% } DOCUMENT 
% Última línea del documento